% Generated by wsMath on Tue Aug 28 2018 20:13:52 GMT+0200 (CEST) (c) J. Mulet (2018)

\documentclass[a4paper]{article},
\usepackage{geometry},
\geometry{a4paper, total={170mm,257mm}, left=20mm, top=20mm},
\usepackage{tasks},
\usepackage[utf8]{inputenc},
\usepackage[T1]{fontenc},
\usepackage{enumitem},
\usepackage{amsmath},
\usepackage{eurosym},  
\usepackage{xcolor},
\definecolor{BLAUCLAR}{RGB}{240,240,255},
\definecolor{MORAT}{RGB}{240,230,255}\n",
\begin{document},      

\begin{center}\large \textbf{ \color{blue} Feina recomanada pels alumnes que han de cursar Matemàtiques Acadèmiques a 4t d'ESO} \end{center}
\vspace{0.5cm}

\fcolorbox{blue}{BLAUCLAR}{ \parbox{0.93\textwidth}{INSTRUCCIONS: Imprimiu aquest dossier i realizeu les activitats proposades. Aquesta feina es presentarà al professor del proper curs dins la primera setmana de classe. La realització correcta d'aquesta tasca serà valorada com a nota de la 1a avaluació. 
 AJUDA: Si necessitau ajuda podeu consultar els apunts o el llibre de 3r d'ESO Matemàtiques i els recursos penjats al curs https://piworld.es}}
\vspace{0.5cm}


 {\small \textbf{Reference}   45lnf9pujiq8r40y  /  553 .} \textbf{Nom i llinatges:}  
          \dotfill
  \section{Radicals}
      \begin{enumerate}[resume]
    \item Escriu les potències en forma d'arrel i viceversa
    \begin{tasks}(2)
      \task $ \sqrt[4]{9}={}$
      \task $ \sqrt[4]{4}={}$
      \task $ \sqrt[3]{9}={}$
      \task $ \sqrt[3]{3}={}$
      \task $2^{\frac{2}{3}}={}$
      \task $2^{\frac{1}{4}}={}$
    \end{tasks}
    \item Calcula el valor numèric de les potències
    \begin{tasks}(2)
      \task $8^{-3}={}$
      \task $\left( -5\right)^{-1}={}$
      \task $9^{-3}={}$
      \task $4^{-3}={}$
      \task $7^{4}={}$
      \task $\left( -6\right)^{1}={}$
      \task $\left( -3\right)^{-4}={}$
      \task $\left( -9\right)^{-3}={}$
    \end{tasks}
    \item Redueix a una única potència
    \begin{tasks}(1)
      \task $\left[2^{-2} \right]^{2} \cdot 2^{-3} : 2^{-3}={\,}$
      \task $10^{3} \cdot 10^{-1}\cdot 10^{-3} : 10^{-1}={\,}$
      \task $\left(-8\right)^{-2} \cdot \left(-8\right)^{0}\cdot \left(-8\right)^{-3} : \left(-8\right)^{3}={\,}$
      \task $\left(-6\right)^{-4} \cdot \left(-6\right)^{4}\cdot \left(-6\right) : \left(-6\right)^{0}={\,}$
      \task $\left[4^{-3} : \left( 1 \right)^{-4} \right]^{-2} \cdot \left( 4 \right)^{3} ={\,}$
      \task $\left[ \dfrac{\left(-8\right)^{-2}\cdot \left( 1 \right)^{-2}}{\left( 64 \right)^{4}} \right]^{2}={\,}$
      \task $\left(-4\right)^{1} \cdot \left( 1 \right)^{3} \cdot \left(-4\right)^{4} : \left( 16 \right)^{4}={\,}$
      \task $\dfrac{ \left(\left( 100 \right)^{-4}\right)^{2} \cdot \left( 1 \right)^{0} : 10^{-2 }}{10^{3}}={\,}$
    \end{tasks}
\par \noindent \vspace{0.25cm} \fcolorbox{purple}{MORAT}{ \parbox{0.88\textwidth}{A vegades és possible simplificar una arrel traient factors defora d'ella. Per això, cal descomposar el radicand en factors primers. Després, tot els factors que estan elevants a l'índex poden sortir davant l'arrel. Exemple: $\sqrt[3]{250}  = \sqrt[3]{5^3 \cdot 2} = 5 \sqrt[3]{2}$}}
\vspace{0.25cm}


    \item Treu factors i simplifica els radicals si és possible
    \begin{tasks}(2)
      \task $4  \sqrt[5]{2^{17} \cdot 3^{6} \cdot 5^{6}}{}={}$
      \task $2  \sqrt[3]{2^{11} \cdot 3^{2} \cdot 5^{3}}{}={}$
      \task $2  \sqrt[3]{2 \cdot 3^{4}}{}={}$
      \task $-10  \sqrt[4]{2^{9} \cdot 5}{}={}$
    \end{tasks}
    \item Opera els *radicals* [ expressant-los prèviament en forma de potència i operant les potències ]
    \begin{tasks}(2)
      \task $ \sqrt{4} \cdot  \sqrt[3]{4}$
      \task $\dfrac{ \sqrt[3]{3} }{  \sqrt{3}}$
      \task $\dfrac{ \sqrt{3} \cdot  \sqrt[4]{3}}{ \sqrt[3]{3}}$
      \task $\dfrac{ \sqrt{2} \cdot  \sqrt[4]{2}}{ \sqrt[3]{2}}$
      \task $\dfrac{ \sqrt{n^{5}} \cdot  \sqrt[4]{n^{5}}}{ \sqrt[3]{n^{5}}}$
      \task $\dfrac{ \sqrt[3]{y^{2}} }{  \sqrt{y^{2}}}$
      \task $\dfrac{ \sqrt{t} }{  \sqrt[3]{t}}$
      \task $\dfrac{ \sqrt{m^{3}} }{  \sqrt[3]{m^{3}}}$
    \end{tasks}
     \end{enumerate}

  \section{Polinomis}
      \begin{enumerate}[resume]
    \item Divideix aquests polinomis utilitzant la regla de Ruffini
    \begin{tasks}(2)
      \task $\left(-10 x^{2} -5 x + 2\right) : \left( x + 5\right)$
      \task $\left(-5 x^{2} + 2 x + 7\right) : \left( x + 2\right)$
      \task $\left(2 x^{3} + 2 x^{2} -10 x -5\right) : \left( x + 7\right)$
      \task $\left(-4 x^{5} +  x^{4} - x^{3} + 2 x^{2}  \right) : \left( x + 9\right)$
    \end{tasks}
    \item Divideix els polinomis
    \begin{tasks}(1)
      \task $\left(2 x^{3} + 3 x^{2} + 2 x -6\right) : \left( x^{2} + 2 x + 2\right)$
      \task $\left( x^{3} -6 x^{2} + 4 x + 3\right) : \left( x^{2} -5 x -1\right)$
      \task $\left(-6 x^{3} + 4 x^{2}  + 6\right) : \left(-3 x^{2} - x -1\right)$
      \task! $\left(15 x^{3} -14 x^{2} -11 x + 8\right) : \left(-5 x^{2} -2 x + 1\right)$
    \end{tasks}
    \item Desenvolupa les identitats notables.
    \begin{tasks}(2)
      \task $\left(a + 3 z\right)^2 = {}$
      \task $\left(x + 5 t\right)^2 = {}$
      \task $\left(y^{4} - 4 z^{4}\right)^2 = {}$
      \task $\left(5 x^{4} + y^{4}\right) \cdot \left(5 x^{4} - y^{4}\right) = {}$
      \task $\left(5 y^{5} + 4 t^{5}\right)^2 = {}$
      \task $\left(\frac{9}{5} a^{3} + 4 t^{3}\right)^2 = {}$
    \end{tasks}
    \item Escriu, si és possible, aquests polinomis com una identitat notable.
    \begin{tasks}(2)
      \task $x^{2} -y^{2} = {}$
      \task $y^{2} +4 t y +4 t^{2} = {}$
      \task $25 y^{4} +10 x^{2} y^{2} +x^{4} = {}$
      \task $25 x^{6} -30 x^{3} y^{3} +9 y^{6} = {}$
    \end{tasks}
    \item Extreu factor comú dels polinomis
    \begin{tasks}(2)
      \task $-25 a^{3} y^{4} +25 a^{3} y^{3} -25 a^{3} y^{2} = {}$
      \task $10 t^{5} x^{6} -6 t^{5} x^{5} -8 t^{5} x^{4} = {}$
      \task $25 t^{5} x^{3} -5 t^{5} x^{2} -25 t^{5} x = {}$
      \task $-t^{3} x^{3} +3 t^{3} x^{2} +t^{3} x = {}$
    \end{tasks}
     \end{enumerate}

  \section{Equacions}
      \begin{enumerate}[resume]
    \item Resol aquestes equacions de segon grau
    \begin{tasks}(2)
      \task $ x^{2} + 3 x + 2 =0$
      \task $ x^{2} +  x -6 =0$
      \task $5 x^{2} + 25 x + 29 =\left(2 x + 5\right)^2$
      \task $5 x^{2} -17 x + 48 =\left(-2 x + 6\right)^2$
    \end{tasks}
    \item Resol aquestes equacions biquadrades (Recorda a aplicar el canvi $t=x^2$)
    \begin{tasks}(1)
      \task $ x^{4}  -5 x^{2}  -36 =0$
      \task $ x^{4}  + 15 x^{2}  -16 =0$
      \task! $- x^{3} -3 x^{2} + 5 x -5 =- x^{4} - x^{3} -8 x^{2} + 5 x -9$
      \task! $ x^{4} + 3 x^{3} -6 x^{2} -6 x -44 =3 x^{3} - x^{2} -6 x -8$
    \end{tasks}
    \item Resol aquestes equacions factoritzades
    \begin{tasks}(2)
      \task $\left( x -2\right) \cdot \left( x  + 1\right) =0$
      \task $\left( x -2\right) \cdot \left( x -4\right) =0$
      \task $ x  \cdot \left( x -2\right) \cdot \left( x  + 3\right)^{2} \cdot \left( x  + 1\right) =0$
      \task $ x  \cdot \left( x  + 4\right) \cdot \left( x  + 3\right) =0$
    \end{tasks}
\par \noindent \vspace{0.25cm} \fcolorbox{purple}{MORAT}{ \parbox{0.88\textwidth}{Per resoldre una equació polinòmica (de grau superior a 2): 

    1. Intentam treure factor comú, 
    2. Miram si identificam alguna identitat notable, 
    3. Si el grau és 3 o més, caldrà fer Ruffini.

VIDEO 52: Equacions polinòmiques}}
\vspace{0.25cm}


    \item Resol aquestes equacions polinòmiques
    \begin{tasks}(2)
      \task $ x^{2} -4 x + 3 =0$
      \task $ x^{2} -4 x  =0$
      \task $ x^{3} -3 x^{2} -16 x + 48 =0$
      \task $ x^{3} + 6 x^{2}  -32 =0$
    \end{tasks}
    \item Resol aquestes sistemes d'equacions
    \begin{tasks}(1)
      \task $\left\{ \begin{array}{ll}  x + y&=-1\\-8 x-4 y&=0\end{array} \right.$
      \task $\left\{ \begin{array}{ll}  x + 8 y&=-70\\4 x-8 y&=40\end{array} \right.$
      \task! $\left\{ \begin{array}{ll}  5\,  x-5\,  y + 4 \,  (x-y)&=4\,  x-4\,  y+31 + 4 \,  (x-3y )\\-4\,  x-15\,  y + 2 \,  (x-y)&=2\,  x-2\,  y-45 + 3 \,  (x-3y )\end{array} \right.$
      \task! $\left\{ \begin{array}{ll}  6\,  x-16\,  y + 2 \,  (x-y)&=2\,  x-2\,  y+47 + 5 \,  (x-2y )\\-2\,  x-14\,  y + 3 \,  (x-y)&=3\,  x-3\,  y+80 + 3 \,  (x-3y )\end{array} \right.$
    \end{tasks}
    \item Determinau el perímetre d'un triangle equilàter sabent que té una àrea de 158 cm$^2$.
    \item Determinau el perímetre d'un triangle equilàter sabent que té una àrea de 438 cm$^2$.
    \item Els tres costats d'un triangle rectangle són proporcionals als números 88, 105, 137. Calcula la longitud de cada costat sabent que l'àrea del triangle és 462000 m$^2$.
    \item  Un pastor diu a un altre pastor: Dóna'm 3 ovelles, i així en tindré el doble que tu. I
                    l'altre li contesta: Dóna-me'n tú 3 ovelles, i així en tindrem tots dos igual. Quantes ovelles
                    té cada pastor?
    \item Si es suma 7 al numerador i al denominador d'una determinada fracció, s'obté la fracció
                    $\frac{2}{3}$. Si en lloc de sumar 7 es resta 3 al numerador i al denominador, s'obté la fracció
                    $\frac{1}{4}$. Trobeu aquesta fracció.
    \item Un orfebre rep l'encàrrec de confeccionar un trofeu, en or i en plata, per a un
                    campionat esportiu. Una vegada realitzat, resulta un pes de 1770 grams, i un cost de 8239 \euro{} .
                    Quina quantitat ha utilitzat de cada preciós de metall, si l'or es ven 7.70 \euro{} /gram i la plata
                    per 2.80 \euro{} /gram?
     \end{enumerate}

  \section{Funcions}
      \begin{enumerate}[resume]
    \item Representa aquestes funcions lineals
    \begin{tasks}(2)
      \task $y = \frac{7}{3}\,x -\frac{1}{8}$
      \task $y = 4\,x+ 6$
      \task $y = \frac{5}{4}\,x -\frac{3}{2}$
      \task $y = \frac{7}{8}\,x -1$
    \end{tasks}
    \item Calcula el vèrtex i representa aquestes paràboles
    \begin{tasks}(2)
      \task $y = -x^2 -8\,x+ 2$
      \task $y = -x^2 -8\,x -6$
      \task $y = x^2 -6\,x -7$
      \task $y = x^2 -8\,x+ 9$
    \end{tasks}
     \end{enumerate}

  \section*{Respostes}
  \begin{enumerate}
    \item 
    \begin{tasks}(2)
      \task $3^{\frac{1}{2}}$
    \end{tasks}
    \item 
    \begin{tasks}(2)
      \task $0.001953125$
    \end{tasks}
    \item 
    \begin{tasks}(2)
      \task $2^{-4} = \dfrac{1}{2^{4}}$
    \end{tasks}
    \item 
    \begin{tasks}(2)
    \end{tasks}
    \item 
    \begin{tasks}(2)
      \task $480  \sqrt[5]{60}$
    \end{tasks}
    \item 
    \begin{tasks}(2)
      \task $2  \sqrt[3]{4}$
    \end{tasks}
    \item 
    \begin{tasks}(2)
      \task $Q(x)=-10 x + 45$; $R=-223$ 
    \end{tasks}
    \item 
    \begin{tasks}(2)
      \task $Q(x)=2 x -1$; $R=-4$ 
    \end{tasks}
    \item 
    \begin{tasks}(2)
      \task $a^{2} +6 a z +9 z^{2}$ 
    \end{tasks}
    \item 
    \begin{tasks}(2)
      \task $\left(x + y\right) \cdot \left(x - y\right)$ 
    \end{tasks}
    \item 
    \begin{tasks}(2)
      \task $5 a^{3} y^{2} \cdot \left(-5 y^{2} +5 y -5\right)$ 
    \end{tasks}
    \item 
    \begin{tasks}(2)
      \task $x=-1$; $x=-2$
    \end{tasks}
    \item 
    \begin{tasks}(2)
      \task $x=-3$; $x=3$
    \end{tasks}
    \item 
    \begin{tasks}(2)
      \task $x=2$; $x=-1$
    \end{tasks}
    \item 
    \begin{tasks}(2)
    \end{tasks}
    \item 
    \begin{tasks}(2)
      \task $x=3$; $x=1$
    \end{tasks}
    \item 
    \begin{tasks}(2)
      \task $ \left(1, \quad -2 \right)$
    \end{tasks}
    \item 
    \begin{tasks}(2)
      \task El perímetre és 57.31 cm
    \end{tasks}
    \item 
    \begin{tasks}(2)
      \task El perímetre és 95.41 cm
    \end{tasks}
    \item 
    \begin{tasks}(2)
      \task Costats 880 m, 1050 m, 1370
    \end{tasks}
    \item 
    \begin{tasks}(2)
      \task El primer té 21 i el segon 15 ovelles.
    \end{tasks}
    \item 
    \begin{tasks}(2)
      \task La fracció és 5/11
    \end{tasks}
    \item 
    \begin{tasks}(2)
      \task 670 g d'or i 1100 g de plata
    \end{tasks}
    \item 
    \begin{tasks}(2)
      \task null
    \end{tasks}
    \item 
    \begin{tasks}(2)
      \task null
    \end{tasks}
  \end{enumerate}
\end{document}